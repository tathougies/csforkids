\week[description={Last week we discussed information. This week we start to understand that combining information with decision making is the foundation on which we can build complex behaviors.}]{Making Decisions}

\section{Numbers All the Way Down}

Last week, we discussed how computers store numbers using switches that are either on or
off. \hint{See \prettyref{sec:binary-numbers} for a review.}

Everything on a computer is stored as numbers. Even the words we saw last class are stored using
numbers. Every letter in a word (or a \term{string} as Python calls it) is assigned a number. We
call these numbers the \term{encoding} of the string.\hint{Python will sometimes abbreviate \term{string} as \code{str}.}

Most modern computers today use the \term{Unicode} standard for encoding strings. Unicode is a
\emph{huge} dictionary mapping every character possibly known to humans to a number assigned by an
international committee.\didyouknow{As of the latest Unicode standard, there are over 297,334
  characters! That's a lot of letters!} Before Unicode standardized the mapping of characters,
computer systems would often use their own custom standard or a standard particular to individual
countries and languages. Reading files created using one standard on a computer designed to work
with another standard could result in garbage data. Another common way to write strings as numbers
is called \term{ASCII}, which stands for the \emph{American Standard Code for Information
Interchange}. ASCII covers most characters used in English and other languages that use the Latin
alphabet. It's also compatible with Unicode -- Unicode maps every Latin letter to the same ASCII
character.

It may seem crazy that each letter has a number that goes along with it, but I can prove it to you. Try this in Python:

\hint{If you want to see a new kind of error called a \code{ValueError}, try\\
\code{ord('hello')}!}
\begin{replbox}
>>> ord('5')<ENTER>
53
>>> ord('Z')<ENTER>
90
>>> ord('>')<ENTER>
62
>>> ord('\'')<ENTER>(*@ \tikzmarknode{python escapes node}{} @*)
39
>>> ord('h')<ENTER>
104
\end{replbox}
\makemargmark{python escapes node}{\notestyle\hintnote Python words are delimited by single quotation marks (\code{'}). But what if you want to talk about the word consisting of just the single quote character itself or a word containing a single quote character (like an Irish name with \enquote{O'})? In order to tell Python that you want a \code{'} treated as just the letter and not part of the Python rules, you use the \code{\textbackslash} character in front of it. This word \code{'\textbackslash''} is just the word consisting of a single single-quote letter.}
%}


You can also go from a number \emph{to} a word.

\hint{Try all kinds of numbers here, and see what you get!}
\begin{replbox}
>>> chr(76)<ENTER>
'L'
>>> chr(53)<ENTER>
'5'
>>> chr(54)<ENTER>
'6'
>>> chr(233)<ENTER>
'é'
>>> chr(1103)<ENTER>
'я'
>>> chr(26159)<ENTER>
'是'
>>> chr(22398)<ENTER>
'学'(*@\tikzmarknode{chinese xie}{}@*)
>>> chr(128512)<ENTER>
'(*@{\EmojiFont 😀}@*)'
\end{replbox}
\makemargmark{chinese xie}{\hintnote \enquote{学}(\emph{xi\`e}) means to learn in Mandarin. That's what we're doing right now, and you just learned something new!}

Lest we forget that it's still all just ones and zeros, you can use the \code{bin} rule that we
learned about last week to see the binary representation of these numbers.

\begin{replbox}
>>> bin(ord('5'))<ENTER>
'0b100111'(*@\tikzmarknode{single quotes around binary numbers}{}@*)
>>> bin(ord('Z'))<ENTER>
'0b1011010'
>>> bin(ord('\''))<ENTER>
'0b100111'
\end{replbox}
\makemargmark{single quotes around binary numbers}{\notestyle\huhnote{I thought the single quotes meant words. Why do we keep seeing them when we ask for binary numbers?}\\
Because Python is friendly, it always shows numbers as decimal, even though they are stored as binary on the computer. The \code{bin} rule actually creates a word, or string, that represents the binary representation. This lets you see it here.}

See it's just numbers all the way down!

\begin{BigIdeaBox}
  All information inside a modern computer is stored as numbers, even words!
\end{BigIdeaBox}
